\documentclass[13pt, a4paper, twoside]{article}
\usepackage[utf8]{inputenc}
\usepackage{geometry}
\usepackage[czech]{babel}
\usepackage{enumitem}
\usepackage{fancyhdr}
\usepackage{amsmath}
\usepackage{mathtools}
\usepackage{float}
\usepackage{setspace}
\usepackage{multicol}
\usepackage{graphicx}
\geometry{legalpaper, margin=1in}
\pagestyle{fancy}
\lhead{\Large Amminkomplexy přechodných kovů}
\rhead{\large Symposion 2023}
\begin{document}
\large \onehalfspacing
\subsection*{Úvod do komplexní problematiky koordinačních sloučenin}
Komplexy (koordinační sloučeniny) jsou molekuly nebo ionty, v nichž na atom nebo ion (\textbf{centrální částici}) jsou donor-akceptorovou vazbou vázány jiné molekuly,
ionty (\textbf{ligandy}). Centrální částice/atom (zkr. CA) je většinou přechodný kov, který v komplexu vystupuje jako \textbf{akceptor volného el. páru},
zatím co ligand poskytuje do vazby volný el. pár a vystupuje tedy jako \textbf{donor volného el. páru}.

\begin{align*}
    [CA(ligandy)]
\end{align*}

To, že jsou atomy nebo ionty přechodných kovů vhodnými centrálními částicemi, není náhoda. Elektronová konfigurace jejich valenční vrstvy je tvořena převážně elektrony \\
$(n-1)d$ orbitalu, což umožňuje, aby při tvorbě koordinačně-kovalentní vazby přijímaly volné elektronové páry do orbitalů $ns$, $np$ a případně do $nd$.
Vhodnými ligandy jsou na druhé straně částice, které disponují volnými el. páry. Příklady ligandů: $H_2O$, $OH^-$, $CN^-$, $NH_3$, $F^-$, $Cl^-$.

Při rospouštění solí vhodných kovů ve vodě dochází ke vzniku tzv. aquakomplexů, ve kterých jako ligand (donor vol. el. páru) vystupuje obyčejná molekula vody.
Tyto molekuly vody vázány na CA (akceptor vol. el. páru) mohou být následně nahrzovány jinými ligandy, které vytvoří společně s CA \textbf{stabilnější komplexní částici}. 
\emph{Tento proces je spojen, jak s uvolněním energie ve formě tepla $\Delta H < 0$, tak se zvýšením entropie systému $\Delta S > 0$. Ze základních termodynamických principů
vyplývá, že tento proces je pro daný systém \uv{výhodný}.}

Stabilita komplexu je kvantitativně vyjádřena tzv. \textbf{konstantou stability komplexu}, $K_k$. Čím větší je hodnota $K_k$, tím je příslušný komplex stabilnější.

V tomto praktiku se seznámíte s amminkomplexy mědi a niklu, které jsou právě stabilnější než jejich aquakomplexy. Nyní si ukážeme vznik amminkomplexu mědi.\\
Při rozpouštění $CuSO_4$ ve vodě dochází ke vzniku světle modrého aquakomplexu mědi.
\begin{align*}
    &\text{\emph{Nejprve dochází k elektrolytické disociaci}}\\
    &CuSO_4 \xrightarrow[]{H_2O} Cu^{2+} + SO_4^{2-}\\
    &\text{\emph{Vznik aquakomplexu}}\\
    &Cu^{2+} + 4H_2O \to [Cu(H_2O)_4]^{2+}\\
    &\text{Název komplexní částice: \dots\dots\dots\dots}
\end{align*}
K takto připravenému roztoku je přidán amoniak v podobě čpavkové vody. Jelikož je amminkomplex mědi stabilnější než aquakomplex, tak dochází k tomu, že
jsou molekuly vody substituovány molekulami amoniaku za vzniku stabilnějšího komplexu.

\begin{align*}
    &[Cu(H_2O)_4]^{2+} + 4NH_3 \to [Cu(NH_3)_4]^{2+} + 4H_2O\\
    &\text{Název komplexní částice: \dots\dots\dots\dots}
\end{align*}

V průběhu reakce pozorujeme barevné změny roztoku, kdy jako meziprodukt vzniká sraženina $Cu(OH)_2 \downarrow$, která se následně rozpouští v nadbytku amoniaku. Výsledné tmavě modré
zbarvení roztoku způsobuje vzniklý komplexní kationt $[Cu(NH_3)_4]^{2+}$. Z tohoto experimentu lze vyvodit, že vzniklý amminkomplex (komplexní kationt tetraamminměďnatý) má
vyšší $K_k$ než aquakomplex (komplexní kationt tetraaquameďnatý)

V rámci tohoto praktika si připravíte již výše zmíněné komplexy mědi a niklu, kde jako ligand vystupuje ethylendiamin (ethan-1,2-diamin zkr. $en$). Navázání \emph{en} na měď, či nikl 
je doprovázeno ještě větším uvolněním energie než navázání amoniaku. Z toho lze vyvodit, že $K_k$ těchto komplexů bude mít vyšší hodnotu.

\begin{center}
\begin{tabular}{|c|}
    \hline
    \emph{pozn. všechna procentualní složení roztoků jsou hmotnostní procenta}\\
    \hline
\end{tabular}
\end{center}

\subsection*{1. Síran bis(ethylendiamin)měďnatý $[Cu(en)_2]SO_4$}
\subsubsection*{Příprava}
Síran bis(ethylendiamin)měďnatý lze připravit reakcí síranu měďnatého se stechiometrickým množstvím ethylendiaminu. 
Ethylendiamin se váže bidentátně tj. prostřednictvím obou atomů dusíku.
Reakci lze zjednodušeně napsat jako:
\begin{align*}
    CuSO_4 + 2H_2N-CH_2-CH_2-NH_2 \to [Cu(en)_2]SO_4
\end{align*}

\begin{figure}[H]
    \centering
    \includegraphics*[width=2.5in]{copperII.png}
\end{figure}

\subsubsection*{Vlastnosti $[Cu(en)_2]SO_4$}
Temně modrý krystalický prášek, rozpustný ve vodě, nerozpustný v organických rozpouštědlech.

\subsubsection*{Postup}
Do kádinky odvažte 5 g modré skalice ($CuSO_4 \cdot 5H_2O$) a přidejte tolik vody, aby vznikl 20\%
roztok. Do kádinky vložte magnetické míchadlo a ponechte míchat na magnetické míchačce do úplného rozpuštění. Následně za stálého míchání pomalu přikapejte 50\% 
roztok ethylendiaminu (2,2 ekvivalentu vůči mědi). Tvoří se tmavě modrý roztok cílového komplexu. K takto připravenému roztoku pomalu přikapejte za míchání 90 ml 
ethanolu a směs ponechte 5 minut v klidu pro dokončení krystalizace. Vyloučený produkt odsajte na fritě. Po odsátí roztoku naplňte fritu ethanolem, produkt v něm 
dobře rozmíchejte a kapalnou fázi opět odsajte. Produkt sušte na fritě prosáváním vzduchem. Suchý produkt vysypte na předem zvážené hodinové sklo a zvažte. 
Vypočtěte teoretický výtěžek a procentuální výtěžek preparace.

\subsection*{2. Chlorid tris(ethylendiamin)nikelnatý $[Ni(en)_3]Cl_2 \cdot 2H_2O$}
\subsubsection*{Příprava}
Dihydrát chloridu tris(ethylendiamin)nikelnatého lze připravit reakcí chloridu nikelnatého s nadbytkem ethylendiaminu. Ethylendiamin se váže bidentátně.


\begin{align*}
    NiCl2 + 3H_2N-CH_2-CH_2-NH_2 \to [Ni(en)_3]Cl_2
\end{align*}

\begin{figure}[H]
    \centering
    \includegraphics*[width=2.5in]{nickelII.png}
\end{figure}

\subsubsection*{Vlastnosti $[Ni(en)_3]Cl_2 \cdot 2H_2O$}
Fialový krystalický prášek rozpustný ve vodě, nerozpustný v acetonu a nepolárních organických rozpouštědlech.

\subsubsection*{Postup}
Do kádinky odvažte 5 g hexahydrátu chloridu nikelnatého ($NiCl_2 \cdot 6H_2O$) a přidejte tolik vody, aby vznikl 20\% roztok. Do kádinky vložte magnetické míchadlo
a ponechte míchat na magnetické míchačce do úplného rozpuštění. Následně za stálého míchání pomalu přikapejte 50\% roztok ethylendiaminu
(6 ekvivalentů vůči niklu). Tvoří se fialový roztok cílového komplexu. K takto připravenému roztoku pomalu přikapejte za míchání 100 ml acetonu a směs
ponechte 5 minut v klidu pro dokončení krystalizace. Vyloučený produkt odsajte na fritě. Po odsátí roztoku naplňte fritu acetonem, produkt v něm dobře 
rozmíchejte a kapalnou fázi opět odsajte. Produkt sušte na fritě prosáváním vzduchem. Suchý produkt vysypte na předem zvážené hodinové sklo a zvažte.
Vypočtěte teoretický výtěžek a procentuální výtěžek preparace.

\subsection*{3. Síran tris(ethylendiamin)měďnatý $[Cu(en)_3]SO_4$}
\Large \emph{* Úkol č. 3 pouze pro ty nejlepší a nejrychlejší}
\large
\subsubsection*{Příprava}
Síran tris(ethylendiamin)měďnatý lze připravit reakcí síranu měďnatého s nadbytkem ethylendiaminu. Ethylendiamin se váže bidentátně.

\begin{align*}
    CuSO_4 + 3H_2N-CH_2-CH_2-NH_2 \to [Cu(en)_3]SO_4
\end{align*}

\begin{figure}[H]
    \centering
    \includegraphics*[width=2.5in]{copper.png}
\end{figure}

\subsubsection*{Vlastnosti $[Cu(en)_3]SO_4$}
Modré krystaly, špatně rozpustné ve vodě. Při promývání vodou či alkoholy dochází k odštěpení jedné molekuly ethylendiaminu za vzniku kationtu bis(ethylendiamin)měďnatého.

\subsubsection*{Postup}
Do kádinky odvažte 5 g modré skalice ($CuSO_4 \cdot 5H_2O$) a přidejte tolik vody, aby vznikl 20\% roztok. Do kádinky vložte magnetické míchadlo a ponechte míchat
na magnetické míchačce do úplného rozpuštění. Následně za stálého míchání pomalu přikapejte 50\%roztok ethylendiaminu (12 ekvivalentů vůči mědi).
Tvoří se tmavě modrý roztok. Následně pomalu za stálého míchání pomalu přikapejte 75 ml ethanolu, přičemž vzniká světlemodrá suspenze.
Směs ponechte 5 minut v klidu pro dokončení krystalizace. Vyloučený produkt odsajte na fritě. Po odsátí roztoku naplňte fritu ethanolem, produkt v
něm dobře rozmíchejte a kapalnou fázi opět odsajte. Produkt sušte na fritě prosáváním vzduchem. Suchý produkt vysypte na předem zvážené hodinové sklo a zvažte.
Vypočtěte teoretický výtěžek a procentuální výtěžek preparace.

\subsection*{Hodnoty a výpočty}
\subsubsection*{\uv{Látkové množství} $n$}
Látkové množství, \emph{n}, má jednotku \emph{mol}\footnote[1]{Jedna ze základních jednotek SI}
\begin{align*}
    n = \frac{m}{M_m}
\end{align*}
Kde $m$ je skutečná hmotnost látky v gramech a  $M_m$ je molární hmotnost (naleznete v tabulkách) v $g\cdot mol^{-1}$

\subsubsection*{Hmotnostní zlomek (procento) $w_{(a)}$}
Jeden ze způsobů vyjádření poměrného zastoupení složek ve směsi (roztoku). Jedná se o tzv. bezrozměrnou veličinu tj. nemá jednotku.
\begin{align*}
    w_{(a)}=\frac{m_{(a)}}{m_\odot}
\end{align*}
Kde $m_{(a)}$ je hmostnost dané složky a $m_{\odot}$ je hmotnost roztoku (směsi)

\subsection*{Tabulka $M_m$}
\begin{center}
    \begin{tabular}{|c|c|}
        \hline
        Látka & $M_m \; (g\cdot mol^{-1})$\\
        \hline
        Ethylendiamin & 60,1\\
        $CuSO_4\cdot 5H_2O$ & 159,6\\
        $NiCl_2\cdot 6H_2O$ & 237.7\\
        \hline
    \end{tabular}
\end{center}






\end{document}